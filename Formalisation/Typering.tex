\section{Typeringsregels}

$\Xi$ is de typeringscontext en houdt bij welke functie welk type heeft op een globaal niveau.
Expressies worden ge\"evalueerd binnen een omgeving $\xi$, welke de types (en waardes) van lokale variabelen bijhoudt.
$\Gamma$ houdt de subtyperingsrelatie bij. $\theta$ weet welke clause van welk
type is.
Wanneer deze tabellen niet nodig zijn, worden ze niet vermeld in de
typeringsregels.
\syntax{De grijze text} stelt een ingegeven lijn code voor, \code{syntax in het vet} duiden op variabele namen.
\newline



\subsection{Typering van constructoren}


\infruleC{$\Xi$, \syntax{data \code{T} = \code{Cons}}}
	{$\Xi \cup $ \code{Cons} : \code{T}}{Simpele constructor}
	
\infruleC{$\Xi$, \syntax{data \code{T} = \code{Cons T0 T1 \ldots}}}
	{$\Xi \cup $ \code{Cons} : \code{T0 \arr T1 \arr \ldots \arr T}}{Constructor met type argumenten}

\infruleC{$\Xi$, \syntax{data \code{T} = \code{Cons0 \ldots} $|$ Cons1 \ldots $|$ \ldots}}
	{$\Xi \cup $ \code{Cons0} : \code{\ldots \arr T} $\cup$ \code{Const1} : \code{\ldots \arr T} $\cup$ \ldots}{Data met meerdere constructoren}

\subsection{Typering van patterns}

Patterns halen een waarde uit elkaar, en slaan die op in een lokale variabele, binnen de tabel $\xi$ (lokale context).
Patterns worden getypeerd met behulp van een type, hier altijd $T$ genoemd.

\infruleC{$\xi$, \syntax{\code{a}}, T}{$\xi \cup $ \code{a} : T}
	{Variabele-toekenning}
	
\infruleC{$\xi$, $\Gamma \vdash $ \code{Cons} $\in$ \code {T}, \syntax{Cons}, T}
{$\xi \cup$ Cons $\in$ T}{Simpele Patternmatch}

\infruleC{$\xi \vdash$ P0 $\in$ T0, P1 $\in$ T1, \ldots , $\Gamma \vdash $
\code{Cons} $\in$ T0 \arr T1 \arr \ldots \arr T, \syntax{\code{Cons
\code{P0} \code{P1} \ldots }}, T}
{$\xi \cup$ \code{Cons} \code{P0} \code{P1} \ldots $\in$ T} {Recursieve
patroontypering}

\subsection{Typering van clauses}
We voeren nog een tabel in: $\theta$. Deze is een 'lokale variabele' die
bijhoudt welke expressie van welk type is.

De eerste regel ziet er vreemd uit, dit is omdat er geen patterns zijn.

\infruleC{$\theta, \Xi \vdash$ \code{expr} : T, \syntax{ = \code{expr}}}
	{$\theta \cup$ \code{ = expr} : T}
	{No patterns-clause}

\infruleC{$\theta \vdash$ \code{P1 \ldots = expr} : T0 \arr T, $\Xi , \xi \vdash$
\code{P0} : T0, \syntax{\code{P0} \code{P1} \ldots = \code{expr}}}
{$\theta \cup $ P0 P1 \ldots = \code{expr} : T} {Recursive-pattern application}


\subsection{Typering van functies}


\infruleC{$\Gamma, \Xi, \xi, \theta \vdash $ clause0 : T0, T1, \ldots, clause1 : T0, T1, \ldots \\
 \syntax{f : T0 \amp T1 \amp \ldots \n clause0 \n clause1 \ldots} }
{$\Gamma \cup $ f : T0 \amp T1 \amp \ldots} {functie-declaratie}

Technisch gezien is $\xi$ en $\theta$ verschillend per clause. Dit is immer de local scope die bij de bijbehorende pattern match hoort.
Elke clause moet een supertype zijn van $f$. Elke clause moet dus een supertype zijn van $T0$,$T1$, \ldots

\subsection{Subtyperingsregels}

Om de regels simpel te houden, houden we geen rekening met de vrije type
variabelen. Deze kunnen immers hernoemd worden wanneer nodig, we gaan ervan uit
dat dit impliciet gebeurt.

\infruleC{$\Gamma \vdash$ e : T, \syntax{instance \code{T} is \code{SuperT}}}
{e : SuperT}{instantie-declaratie}

