\section{Syntax}

Zie de bnf-bestanden voor de volledige syntax met syntactische suiker. 

Hieronder volgt er een kort, involledig overzicht van de belangerijkste regels.

\subsection{Expressies}

\begin{lstlisting}[frame=single]
functionName	::= [a..z][a..zA..Z0..9]
constructorName ::= [A..Z][a..zA..Z0..9]
-- een expressie is een functieoproep met argumenten
-- een functionName kan ook een locale variabele zijn
expression ::= (functionName | constructor) expression* 
		| "(" expression ")"
\end{lstlisting}

Een constructor is bv. \code{True}, \code{False}, maar ook \code{Nil} en
\code{Elem}.

Een expressie is bv \code{Elem True (Elem False Nil)}

\subsection{Types}
\begin{lstlisting}[frame=single]
knownType	::= [A..Z][a..zA..Z0..9]
freeType	::= [a..z][a..zA..Z0..9]
constrainedType	::= "("
simpleType	::= knownType | freeType | "(" type ")" 
appliedType	::= simpleType+	 
type		::= appliedType ("->" type) *
\end{lstlisting}

Types zijn bv. \code{Bool}, \code{Bool \arr Bool}, \code{(a \arr b) \arr
List a \arr List b}

\subsection{Nieuwe types}

\begin{lstlisting}[frame=single]
data	::= "data" knownType freeType* "=" 
	   (constructor type*) ("|" constructor type*)*
cat		::= "cat" knownType freeTypes 
			("\n" functionDeclaration)+
instance	::= "instance" type "is" type
\end{lstlisting}

Voorbeelden hiervan zijn:

\begin{lstlisting}[backgroundcolor=\color{grey}]
data Bool 	= True | False
data List a	= Nil | Elem a (List a)

cat Functor a
     map : (a -> b) -> (functor:Functor) a -> functor b
	
instance List is Functor

\end{lstlisting}


\subsection{Functiedeclaraties en pattern matching}

\begin{lstlisting}[frame=single]
variable::= functionName
pattern	::= variable | constructor | "(" pattern+ ")"
function::= functionName ":" type ("&" type)* "\n" 
	(pattern "=" expression)+
\end{lstlisting}

Examples of functions are:

\begin{lstlisting}[backgroundcolor=\color{grey}]

not	: Bool -> Bool
True	= False
False	= True

and		: Bool -> Bool -> Bool
True True	= True
_ _		= False

map		: (a -> b) -> List a -> List b
_ Nil		= Nil
f (Elem a list)	= Elem (f a) (map f list)

\end{lstlisting}